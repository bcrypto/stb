\section{Генерация псевдослучайных чисел в режиме HMAC}\label{PRNG-HMAC}

\subsection{Функция хэширования}

Используется функция хэширования~$h$,
удовлетворяющая ограничениям HMAC (пункт~\ref{COMMON.Hash}).
%
Функция~$h$ применяется косвенно~--- как композиционный элемент
алгоритма~$\HMAC[h]$.

\subsection{Входные и выходные данные}

Входными данными алгоритма генерации псевдослучайных чисел 
являются натуральное~$n$, ключ~$K\in\{0,1\}^*$ и синхропосылка~$S\in\{0,1\}^*$. 
Число~$n$ определяет количество генерируемых псевдослучайных чисел.

Выходными данными алгоритма является слово~$Y\in\{0,1\}^{2ln}$~---
псевдослучайные числа, полученные на ключе~$K$ при использовании
синхропосылки~$S$. 
Слово~$Y$ записывается в виде
$Y=Y_1\parallel Y_2\parallel\ldots\parallel Y_n$, 
где~$Y_i\in\{0,1\}^{2l}$.

\subsection{Переменные}

Используется переменная $r\in\{0,1\}^{2l}$.

\subsection{Алгоритм}

Генерация псевдослучайных чисел состоит в выполнении следующих шагов:
\begin{enumerate}
\item
$r\leftarrow\HMAC[h](K,S)$.
\item
Для $i=1,2,\ldots,n$ выполнить:
\begin{enumerate}
\item
$Y_i\leftarrow\HMAC[h](K, r\parallel S)$;
\item
$r\leftarrow \HMAC[h](K,r)$.
\end{enumerate}
\item
$Y\leftarrow Y_1\parallel Y_2\parallel\ldots\parallel Y_n$.
\item
Возвратить $Y$.
\end{enumerate}
