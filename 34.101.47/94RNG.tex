\begin{appendix}{Г}{справочное}{Использование генераторов случайных чисел}
\label{RNG}

\mbox{}

Генератор случайных чисел вырабатывает последовательности,
каждый следующий элемент которых статистически и вычислительно 
трудно предсказать по всем предыдущим элементам.
%
Генератор использует один или несколько 
источников случайности и включает средства 
обработки данных от источников. 

В компьютерных системах распространены следующие источники случайности:
\begin{itemize}
\item[--]
физические источники, использующие процессы в физических устройствах
(например, шум в радиоэлектронных приборах);

\item[--]
системные источники, использующие состояния, 
процессы и события операционной системы
(системное время, сетевая активность, прерывания);

\item[--]
источники, основанные на активности операторов
(движения мышью, нажатия клавиш).
\end{itemize}

Предпочтительным является использование физических источников случайности.

Для источника случайности~$S$ проводится оценка энтропии 
(неопределенности, вариабельности) его выходных последовательностей.
Для этого строится вероятностная модель источника~$S$ 
и в рамках этой модели определяется величина~$h$ такая, 
что основная вероятностная масса выходных последовательностей 
длины~$n$ сосредоточена на множестве мощности~$2^{nh}$.
Величина~$h$ называется удельной энтропией на наблюдение.
%
Например, если~$S$ выдает случайные независимые символы 
алфавита~$A$ и вероятность появления символа~$\alpha$ 
равняется~$p_\alpha$, то удельная энтропия рассчитывается по формуле
$$
h=-\sum_{\alpha\in A}p_\alpha\log_2 p_\alpha\quad
(0\cdot\log_2 0=0).
$$

Кроме~$h$ существуют и другие характеристики неопределенности.
Например, в криптографических приложениях используют 
минимальную удельную энтропию~$h_{min}$, которая характеризует 
сложность предсказания самой вероятной выходной последовательности~$S$.
Для описанного выше источника минимальная удельная энтропия
определяется как
$$
h_{min}=\min_{\alpha\in A}(-\log_2 p_\alpha).
$$

Оценка величины~$h$ (или $h_{min}$) является сложной задачей, 
если распределение $\{p_\alpha\}$ известно не полностью,
источник~$S$ не является стационарным,
между выходными символами~$S$ имеются зависимости и в других ситуациях.
%
Для оценки~$h$ могут применяться статистические методы,
основанные на частотах встречаемости в выходных последовательностях
$m$-грамм, а также алгоритмические методы, основанные на коэффициентах 
обратимого или необратимого сжатия выходных 
последовательностей.

Если удельная энтропия $h$ оценена, то можно сделать вывод о том, что для 
надежной генерации секретного ключа длины~$l$ требуется использовать 
не менее $l/h$ наблюдений от источника случайности.

\end{appendix}
